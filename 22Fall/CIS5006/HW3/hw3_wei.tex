

%EX TS-program = pdflatex
% !TEX encoding = UTF-8 Unicode

% This is a simple template for a LaTeX document using the "article" class.
% See "book", "report", "letter" for other types of document.

\documentclass[11pt]{article} % use larger type; default would be 10pt
\usepackage[utf8]{inputenc} % set input encoding (not needed with XeLaTeX)

%%% Examples of Article customizations
% These packages are optional, depending whether you want the features they provide.
% See the LaTeX Companion or other references for full information.
\usepackage{amsmath}
\makeatletter
\renewcommand*\env@matrix[1][*\c@MaxMatrixCols c]{%
	\hskip -\arraycolsep
	\let\@ifnextchar\new@ifnextchar
	\array{#1}}
\makeatother

\newcommand{\norm}[1]{\left\lVert#1\right\rVert}
%%% PAGE DIMENSIONS
\usepackage{geometry} % to change the page dimensions
\usepackage{listings}
\usepackage[dvipsnames]{xcolor}
\usepackage{marvosym}
\geometry{a4paper} % or letterpaper (US) or a5paper or....
% \geometry{margin=2in} % for example, change the margins to 2 inches all round
% \geometry{landscape} % set up the page for landscape
%   read geometry.pdf for detailed page layout information

\usepackage{graphicx} % support the \includegraphics command and options
% \usepackage[parfill]{parskip} % Activate to begin paragraphs with an empty line rather than an indent
\usepackage{amssymb}
\usepackage{natbib}
\usepackage{hyperref}
\usepackage{mathrsfs}
%%% PACKAGES
\usepackage{booktabs} % for much better looking tables
\usepackage{array} % for better arrays (eg matrices) in maths
\usepackage{paralist} % very flexible & customisable lists (eg. enumerate/itemize, etc.)
\usepackage{verbatim} % adds environment for commenting out blocks of text & for better verbatim
\usepackage{subfig} % make it possible to include more than one captioned figure/table in a single float
% These packages are all incorporated in the memoir class to one degree or another...
\usepackage{pgfplots}
%%% HEADERS & FOOTERS
\usepackage{fancyhdr} % This should be set AFTER setting up the page geometry
\pagestyle{fancy} % options: empty , plain , fancy
\renewcommand{\headrulewidth}{0pt} % customise the layout...
\lhead{}\chead{}\rhead{}
\lfoot{}\cfoot{\thepage}\rfoot{}

%%% SECTION TITLE APPEARANCE
\usepackage{sectsty}
\allsectionsfont{\sffamily\mdseries\upshape} % (See the fntguide.pdf for font help)
% (This matches ConTeXt defaults)
\usepackage[thinc]{esdiff}
\usepackage{bbold}
\usepackage{MnSymbol,wasysym}
%%% ToC (table of contents) APPEARANCE
\usepackage[nottoc,notlof,notlot]{tocbibind} % Put the bibliography in the ToC
\usepackage[titles,subfigure]{tocloft} % Alter the style of the Table of Contents
\renewcommand{\cftsecfont}{\rmfamily\mdseries\upshape}
\renewcommand{\cftsecpagefont}{\rmfamily\mdseries\upshape} % No bold!

%%% END Article customizations

%%% The "real" document content comes below...

\title{Homework 3}
\author{Wei Ye\footnote{2nd year PhD student in Economics Department at Fordham University. Email: wye22@fordham.edu}
    \\ CIS5006 - Data Structure}
\date{Due on Oct 16, 2022}
\begin{document}
\maketitle

\section{Stack}
\begin{itemize}
    \item 5.
    
    It's overflow. As the top element has already reached to the max index of this array, and our array is static array. We can't expand the array dynamically, so it's overflow if we push another element.

    Letter: X, Stack.top: 4; overflow: T; underflow = F   items: A, B, C, D, E

    \item 7.\\
    Letter = z; stack.top: 3; overflow = F; underflow = F; items: A,B,X,Y
    
    \item 10.\\
    a
    \item 15. \\ 
    \begin{lstlisting}{language= C++}
        
        ReplaceItem(StackType &stack, ItemType oldItem, ItemType newItem){
            //use newStack to receive the newItem
            stack <ItemType> newStack;
            while(!stack.isEmpty()){
                if(stack.top()==oldItem){ newStack.push(newItem);}
                else {newStack.push(stack.top());}
                stack.pop();
            }
            while(!newStack.isEmpty()){
                stack.push(newStack.top());//put it back
                newStack.pop();
            }
        }
        
    \end{lstlisting}
    Time complexity is still $O(n)$. The drawback here is we increase the space complexity by creating a new stack. 

    \item 16.\\
    \begin{lstlisting}{language=C++}
        //set up the candy container as a stack.
        stack <string> candy;//colors in string
        stack <string> newCandy;
        for(int i = 0; i < candy.size();++i){
            while(candy.top()!= "yellow"){
                newCandy.push(candy.top());
                candy.pop();
            }
            candy.pop();
        }
        for (int i = 0; i < newCandy.size();++i){
            //put candies in originial order except
            //we consumed yellow candies.
            candy.push(newCandy.top());
            newCandy.pop();
        }
    \end{lstlisting}

    \item 18.\\
    \begin{lstlisting}{language = C++}
        Bool Identical(StackType stack1, StackType stack2){
            if (stack1.length()!= stack2.length()){
                return false;
            }
            while(!stack1.isEmpty() &&!stack2.isEmpty()){
                if (stack1.top()==stack2.top()){
                        stack1.pop();
                        stack2.pop();
                } else{
                    return false;
                }
            }
            return true;
        }
    \end{lstlisting}
    
\end{itemize}




\section{Queue}
\begin{itemize}
    \item 22.\\
    \begin{enumerate}[a.]
        \item First, enqueue a couple of elements in the queue. 0, 1, 5. The pop out 0, it becomes 1,5. Then enqueue  16, 0, 4 respectively. It becomes 1,5,16,0,4. Dequeue 1. The queue becomes 5,16,0,4.
        And then it prints items 1, 2, and 3. while loop is for when the enumerate the queue, when the queue is not empty, always dequeue 1. When dequeue 1, it print 1 to show how many item1 have been dequeued.

        cout would be 1 0 4 5 16 0 3

        \item It first forms item2 from the summation of item1 and 1. item2 becomes 5. First, the queue pushes (enqueue) some items. it becomes 5,6,4. then dequeue item 2, becoming 6,4. Assign item1 with new value to 6. then enqueue, the queue is 6,4,4,0.while loop to pop out all the item3. then print item3. 
        In the end, print the new values of item1, 2, and 3.

        cout would be 6 4 6 0 6 5 0
    \end{enumerate}

    \item 26.\\
L, 0, 4, F, F, V, W, X, Y, L.

    \item 29.\\
W, 2, 0, F, F, V  X Y Z (note: there is no char at index 1)


\item 33.\\
The logic is to use new container(queue) tp receive the elements, then put them back. The client can
only get the temp, others are unavailable. Here are the codes.
    \item \begin{lstlisting}{language = C++}
    ItemType Front(QueType &queue){
        ItemType temp;
        QueType newQue;
        //receive the element that's been dequeued from original queue.
        queue.Dequeue(temp);
        newQue.Enqueue(temp);
        ItemType item;
        while(!queue.isEmpty()){
            queue.Dequeue(item);//pop out the rest
            newQue.Enqueue(item);
        }
        //add items back to original queue.
        while(!newQue.isEmpty()){
            newQue(item);//pop out
            queue.Enqueue(item);
        }
        return temp;//return the front here.

    }
    \end{lstlisting}



\item 34.\\

\begin{lstlisting}{language = C++}
    void ReplaceItem(QueType queue, int oldItem, int newItem){
        ItemType temp;
        QueType newQue;
        while(!queue.isEmpty()){
            queue.Dequeue(temp);
            if (temp == oldItem){
                newQue.Enqueue(newItem);
            }else{
                newQue.Enqueue(temp);
            }
        }
        //put the elements of newQue back to queue.
        while(!newQue.isEmpty()){
            newQue.Dequeue(temp);
            queue.Enqueue(temp);
        }
    }
    
\end{lstlisting}

\item 40.\\
\begin{enumerate}
    \item If as a constructor, it's called by default when we implement the class containing the constructor. However, if MakeEmpty() as an operation,
    it's only called when the user does the operation. 
    \item Constructor can initialize private data member in the class. MakeEmpty() can reset private members but data stored in the class can't be changed.
    \item Constructor doesn't return any type of data, but MakeEmpty() can, aka void as return type.
\end{enumerate}

\end{itemize}

\end{document}