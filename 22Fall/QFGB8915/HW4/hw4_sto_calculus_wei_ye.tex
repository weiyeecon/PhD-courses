%EX TS-program = pdflatex
% !TEX encoding = UTF-8 Unicode

% This is a simple template for a LaTeX document using the "article" class.
% See "book", "report", "letter" for other types of document.

\documentclass[11pt]{article} % use larger type; default would be 10pt
\usepackage[utf8]{inputenc} % set input encoding (not needed with XeLaTeX)

%%% Examples of Article customizations
% These packages are optional, depending whether you want the features they provide.
% See the LaTeX Companion or other references for full information.
\usepackage{amsmath}
\makeatletter
\renewcommand*\env@matrix[1][*\c@MaxMatrixCols c]{%
	\hskip -\arraycolsep
	\let\@ifnextchar\new@ifnextchar
	\array{#1}}
\makeatother

\newcommand{\norm}[1]{\left\lVert#1\right\rVert}
%%% PAGE DIMENSIONS
\usepackage{geometry} % to change the page dimensions
\usepackage{listings}
\usepackage[dvipsnames]{xcolor}
\usepackage{marvosym}
\geometry{a4paper} % or letterpaper (US) or a5paper or....
% \geometry{margin=2in} % for example, change the margins to 2 inches all round
% \geometry{landscape} % set up the page for landscape
%   read geometry.pdf for detailed page layout information

\usepackage{graphicx} % support the \includegraphics command and options
% \usepackage[parfill]{parskip} % Activate to begin paragraphs with an empty line rather than an indent
\usepackage{amssymb}
\usepackage{natbib}
\usepackage{hyperref}
\usepackage{mathrsfs}
%%% PACKAGES
\usepackage{booktabs} % for much better looking tables
\usepackage{array} % for better arrays (eg matrices) in maths
\usepackage{paralist} % very flexible & customisable lists (eg. enumerate/itemize, etc.)
\usepackage{verbatim} % adds environment for commenting out blocks of text & for better verbatim
\usepackage{subfig} % make it possible to include more than one captioned figure/table in a single float
% These packages are all incorporated in the memoir class to one degree or another...
\usepackage{pgfplots}
%%% HEADERS & FOOTERS
\usepackage{fancyhdr} % This should be set AFTER setting up the page geometry
\pagestyle{fancy} % options: empty , plain , fancy
\renewcommand{\headrulewidth}{0pt} % customise the layout...
\lhead{}\chead{}\rhead{}
\lfoot{}\cfoot{\thepage}\rfoot{}

%%% SECTION TITLE APPEARANCE
\usepackage{sectsty}
\allsectionsfont{\sffamily\mdseries\upshape} % (See the fntguide.pdf for font help)
% (This matches ConTeXt defaults)
\usepackage[thinc]{esdiff}
\usepackage{bbold}
\usepackage{MnSymbol,wasysym}
%%% ToC (table of contents) APPEARANCE
\usepackage[nottoc,notlof,notlot]{tocbibind} % Put the bibliography in the ToC
\usepackage[titles,subfigure]{tocloft} % Alter the style of the Table of Contents
\renewcommand{\cftsecfont}{\rmfamily\mdseries\upshape}
\renewcommand{\cftsecpagefont}{\rmfamily\mdseries\upshape} % No bold!

%%% END Article customizations

%%% The "real" document content comes below...

\title{Homework 4}
\author{Wei Ye\footnote{2nd year PhD student in Economics Department at Fordham University. Email: wye22@fordham.edu}
    \\ QF8915 - Stochastic Calculus}
\date{Due on Dec 16, 2022}
\begin{document}
\maketitle

\section*{Problem 1}
Apply Ito's formula to express $Y_t = \log(1+(X_t)^2)$ as an Ito process.

\textbf{Solution:}

First, we assume $Y_t = f(X_t) = \log(1+(X_t)^2)$. Thus, $f'(X_t) = \frac{2X_t}{1+X_t^2}$, and $f''(X_t) = \frac{2(1+X_t^2)-2X_t(2X_t)}{(1+X_t^2)^2} = \frac{2(1-X_t^2)}{(1+X_t^2)^2}$


\begin{align*}
    dY_t &= f'(X_t)dX_t + \frac{1}{2}f''(X_t)dX_tdX_t\\
         &= \frac{2X_t}{1+X_t^2}[W(t)dt +W(t)^2dW_t] +\frac{1}{2}\cdot\frac{2(1-X_t^2)}{(1+X_t^2)^2}W(t)^4dt\\
         &= [\frac{2X_t^2}{1+X_t^2}W(t)+\frac{1-X_t^2}{(1+X_t^2)^2}W(t)^4]dt+ \frac{2X_t}{1+X_t^2}[W(t)]^2dW(t)
\end{align*}

Thus:
\begin{equation*}
    Y_t = Y_0+ \int_0^t \frac{2X_s}{1+X_s}W(s)+ \frac{1-X_s^2}{(1+X_s^2)^2}W(s)^4ds + \int_0^t \frac{2X_s}{1+X_s^2}W(s)^2 dW(s)
\end{equation*}

\section*{Problem 2}
Let $Y_t = (1+t)X_t$. Use Ito's formula to find out what stochastic differential equation $Y_t$ satisfies? Identify $Y_t$ as a Brownian Motion.

\textbf{Solution:}

Let $Y_t = f(t,X_t) = (1+t)X_t$. Therefore, $f_t(t,X_t) = X_t$, $f_x(t,X_t) = 1+t$, and $f_{xx} = 0$. 
\begin{align*}
    dY_t &= f_t(t,X_t)dt + f_x(t,X_t)dX_t + \frac{1}{2}f_{xx}dX_t dX_t\\
        &= X_t dt + (1+t)dX_t\\
        &= X_t dt + (1+t)[-\frac{1}{1+t}X_td_t + \frac{1}{1+t}dW(t)]\\
        &= X_t dt- X_tdt +dW_t\\
        &= dW_t
\end{align*}


Thus, we can easily see $Y_t$ is brownian motion.

\section*{Problem 3}
Solve $dX_t = X_tdt + dW(t)$.

\textbf{Solution:}

Set an equation\footnote{For the convenience of our computation, I let $\beta$ in the lecture is equal to -1, only in this way, we can cancel out $X_t$ in later steps.} $f(t,X_t) = e^{-t}X_t$

\begin{align*}
    df(t,X_t) &= -e^{-t}X_tdt + e^{-t}[X_tdt + dW_t]\\
              &= -e^{-t}X_tdt + e^{-t}X_tdt + e^{-t}dW_t\\
              &= e^{-t}dW(t)
\end{align*}
\begin{equation*}
    e^{-t}X_t = X_0 + \int_0^t e^{-u}dW(u)
\end{equation*}
Therefore:
\begin{equation*}
    X_t = e^tX_0 +e^t \int_0^t e^{-u}dW(u)
\end{equation*}
\begin{equation*}
    E(X_t) = e^tX_0
\end{equation*}


\section*{Problem 4}
Solve $dX_t = -X_td_t +e^{-t}dW(t)$

\textbf{Solution:}

Let $f(t,X_t) = e^tX_t$
\begin{align*}
    df(t,X_t) &= e^tX_tdt + e^t[-X_tdt + e^{-t}dW(t)]\\
             &= dW(t)
\end{align*}
Hence,
\begin{equation*}
    e^tX_t = x_0 + \int_0^t dW_t
\end{equation*}
\begin{equation*}
    X_t = e^{-t}X_0 +e^{-t}\int_0^t dW_t
\end{equation*}

\begin{align*}
    E(X_t) &= e^{-t}X_0 +e^{-t}\int_0^t dW_u\\
        &= e^{-t}X_0
\end{align*}

\end{document}