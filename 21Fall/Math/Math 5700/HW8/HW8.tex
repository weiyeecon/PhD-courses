%EX TS-program = pdflatex
% !TEX encoding = UTF-8 Unicode

% This is a simple template for a LaTeX document using the "article" class.
% See "book", "report", "letter" for other types of document.

\documentclass[11pt]{article} % use larger type; default would be 10pt

\usepackage[utf8]{inputenc} % set input encoding (not needed with XeLaTeX)

%%% Examples of Article customizations
% These packages are optional, depending whether you want the features they provide.
% See the LaTeX Companion or other references for full information.
\usepackage{amsmath}
\makeatletter
\renewcommand*\env@matrix[1][*\c@MaxMatrixCols c]{%
	\hskip -\arraycolsep
	\let\@ifnextchar\new@ifnextchar
	\array{#1}}
\makeatother
%%% PAGE DIMENSIONS
\usepackage{geometry} % to change the page dimensions
\usepackage{marvosym}
\geometry{a4paper} % or letterpaper (US) or a5paper or....
% \geometry{margin=2in} % for example, change the margins to 2 inches all round
% \geometry{landscape} % set up the page for landscape
%   read geometry.pdf for detailed page layout information

\usepackage{graphicx} % support the \includegraphics command and options
% \usepackage[parfill]{parskip} % Activate to begin paragraphs with an empty line rather than an indent
\usepackage{amssymb}
%%% PACKAGES
\usepackage{booktabs} % for much better looking tables
\usepackage{array} % for better arrays (eg matrices) in maths
\usepackage{paralist} % very flexible & customisable lists (eg. enumerate/itemize, etc.)
\usepackage{verbatim} % adds environment for commenting out blocks of text & for better verbatim
\usepackage{subfig} % make it possible to include more than one captioned figure/table in a single float
% These packages are all incorporated in the memoir class to one degree or another...
\usepackage{pgfplots}
%%% HEADERS & FOOTERS
\usepackage{fancyhdr} % This should be set AFTER setting up the page geometry
\pagestyle{fancy} % options: empty , plain , fancy
\renewcommand{\headrulewidth}{0pt} % customise the layout...
\lhead{}\chead{}\rhead{}
\lfoot{}\cfoot{\thepage}\rfoot{}

%%% SECTION TITLE APPEARANCE
\usepackage{sectsty}
\allsectionsfont{\sffamily\mdseries\upshape} % (See the fntguide.pdf for font help)
% (This matches ConTeXt defaults)
\usepackage[thinc]{esdiff}
\usepackage{bbold}
\usepackage{MnSymbol,wasysym}
%%% ToC (table of contents) APPEARANCE
\usepackage[nottoc,notlof,notlot]{tocbibind} % Put the bibliography in the ToC
\usepackage[titles,subfigure]{tocloft} % Alter the style of the Table of Contents
\renewcommand{\cftsecfont}{\rmfamily\mdseries\upshape}
\renewcommand{\cftsecpagefont}{\rmfamily\mdseries\upshape} % No bold!

%%% END Article customizations

%%% The "real" document content comes below...

\title{HW8}
\author{Wei Ye\footnote{I worked on my assignment sololy. Email: wye22@fordham.edu}  	\\
	ECON 5700}
\date{Due on August 22, 2020.}


\begin{document}
	\maketitle
	\section{Question 1}
	\textbf{Solution:}
	
\begin{align*}
	\det(A)&=(-1)^5\cdot 2\begin{bmatrix}
		2&-3&1\\
		1&-1&3\\
		-2&1&0
	\end{bmatrix}\\
&=-2[(-1)^4(-2)(-9+1)+(-1)^5(6-1)]\\
&=-2(16-5)\\
&=-22
\end{align*}

\section{Question 2}
\textbf{Solution:}

Because A is upper triangle matrix, thus, its determinant is:
$$\det(A)=2\cdot 3\cdot 1\cdot 5\cdot -1=-30$$	


\section{Question 3}
\textbf{Solution:}

$$AB=\begin{bmatrix}
	12&3\\16&5
\end{bmatrix}$$
$$\det(AB)=12\cdot 5-16\cdot 3=12$$


\section{Question 4}
\textbf{Solution:}

$$A^{-1}=\begin{bmatrix}
	\frac{3}{4}&-\frac{1}{4}\\
	-\frac{1}{2}&\frac{1}{2}
\end{bmatrix}$$
$$\det(A^{-1})=\frac{3}{8}-\frac{1}{8}=\frac{1}{4}$$


\section{Question 5}
\textbf{Solution:}

\begin{align*}
	\begin{bmatrix}
		A-\lambda 
	\end{bmatrix}&=\begin{bmatrix}
		1-\lambda&0&1\\
		0&1-\lambda&1\\
		1&1&-\lambda
	\end{bmatrix}\\
	&=0
\end{align*}

Thus, $\lambda_1=2$, $\lambda_2=-1$, and $\lambda_3=1$.

Thus, the matrix $D=\begin{bmatrix}
	2&0&0\\
	0&-1&0\\
	0&0&1
\end{bmatrix}$

From this, we can obtain the diagonal matrix is :

$$A=\mathcal{P}D\mathcal{P}^{-1}$$

$$
[e_1\ e_2\ e_3]
\begin{bmatrix}
	2&0&0\\
	0&-1&0\\
	0&0&1
\end{bmatrix} [e_1\ e_2\ e_3]^{-1}$$

\section{Question 6}
\textbf{Solution:}
\begin{enumerate}
	\item The first matrix =8
	\item The second matrix is $4\cdot2\cdot\frac{1}{3}\cdot(-1)=-\frac{3}{8}$
	\item The result is -4
	\item The result is 4
	\item the result is $2\cdot (-1)=-2$
	\item The result is 4
\end{enumerate}
\textcolor{blue}{Need to check as I get the lecture note 8. Forgot the specific rules.}

\section{Question 7}
\textbf{Solution:}
\begin{align*}
	A^{10}&=\mathcal{P}D^{10}P\\
	&=[e_1\ e_2]\begin{bmatrix}
		\frac{1}{2}&0\\
		0&2
	\end{bmatrix}[e_1\ e_2]^{-1}\\
	&= [e_1\ e_2]\begin{bmatrix}
		\frac{1}{2^{10}}&0\\
		0&2^{10}
	\end{bmatrix}[e_1\ e_2]^{-1}
\end{align*}


\section{Question 8}
\textbf{Solution:}
Let $A=\begin{bmatrix}
	-1&0\\
	1&0
\end{bmatrix}$, 
\begin{align*}
	\begin{bmatrix}
		-1-\lambda&6\\
		1&-\lambda
	\end{bmatrix}&=-\lambda(-1-\lambda)-6\\
	&=0
\end{align*}
We can obtain the eignvalues of the matrix, $\lambda_1=-3, \lambda_2=2$. $D=\begin{bmatrix}
	-3&0\\
	0&2
\end{bmatrix}$
$$A^{10}=\mathcal{P}D^{10}\mathcal{P}^{-1}=[e_1\ e_2]\begin{bmatrix}
	(-3)^{10}&0\\
	0&2^{10}
\end{bmatrix}[e_1\ e_2]^{-1}$$

\section{Question 9}
\textbf{Solution:}
First, we need to calculate the the eigenvalues of the matrix,
$$\begin{bmatrix}
	1-\lambda&0&k\\
	0&1-\lambda&0\\
	0&0&1-\lambda
\end{bmatrix}=0$$
Thus $\lambda_1=\lambda_2=\lambda_3=1$.
Thus, the algebraic multiplicity is 3. 
$A-\lambda I= \begin{bmatrix}
	0&0&k\\
	0&0&0\\
	0&0&0
\end{bmatrix}$. If  the matrix is diagonaize, it should be that $AM=GM\longrightarrow k=0\in \mathcal{R}$

	
	
	
	
	
	
	
	
	
	
	
	
	
	
	
	
	
	
	
	
	
	
	
	
	
	
	
	
	
	
	
	
	
	
	
	
	
	
	
	
	
	
	
	
	
	
	
	
	
	
	
	
	
	
	
	
	
	
	
	
	
	
	
	
	
	
	
	
	
	
	
	
	
	
	
	
	
	
	
	
	
\end{document}