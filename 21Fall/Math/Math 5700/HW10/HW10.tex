%EX TS-program = pdflatex
% !TEX encoding = UTF-8 Unicode

% This is a simple template for a LaTeX document using the "article" class.
% See "book", "report", "letter" for other types of document.

\documentclass[11pt]{article} % use larger type; default would be 10pt

\usepackage[utf8]{inputenc} % set input encoding (not needed with XeLaTeX)

%%% Examples of Article customizations
% These packages are optional, depending whether you want the features they provide.
% See the LaTeX Companion or other references for full information.
\usepackage{amsmath}
\makeatletter
\renewcommand*\env@matrix[1][*\c@MaxMatrixCols c]{%
	\hskip -\arraycolsep
	\let\@ifnextchar\new@ifnextchar
	\array{#1}}
\makeatother

\newcommand{\norm}[1]{\left\lVert#1\right\rVert}
%%% PAGE DIMENSIONS
\usepackage{geometry} % to change the page dimensions
\usepackage{marvosym}
\geometry{a4paper} % or letterpaper (US) or a5paper or....
% \geometry{margin=2in} % for example, change the margins to 2 inches all round
% \geometry{landscape} % set up the page for landscape
%   read geometry.pdf for detailed page layout information

\usepackage{graphicx} % support the \includegraphics command and options
% \usepackage[parfill]{parskip} % Activate to begin paragraphs with an empty line rather than an indent
\usepackage{amssymb}
%%% PACKAGES
\usepackage{booktabs} % for much better looking tables
\usepackage{array} % for better arrays (eg matrices) in maths
\usepackage{paralist} % very flexible & customisable lists (eg. enumerate/itemize, etc.)
\usepackage{verbatim} % adds environment for commenting out blocks of text & for better verbatim
\usepackage{subfig} % make it possible to include more than one captioned figure/table in a single float
% These packages are all incorporated in the memoir class to one degree or another...
\usepackage{pgfplots}
%%% HEADERS & FOOTERS
\usepackage{fancyhdr} % This should be set AFTER setting up the page geometry
\pagestyle{fancy} % options: empty , plain , fancy
\renewcommand{\headrulewidth}{0pt} % customise the layout...
\lhead{}\chead{}\rhead{}
\lfoot{}\cfoot{\thepage}\rfoot{}

%%% SECTION TITLE APPEARANCE
\usepackage{sectsty}
\allsectionsfont{\sffamily\mdseries\upshape} % (See the fntguide.pdf for font help)
% (This matches ConTeXt defaults)
\usepackage[thinc]{esdiff}
\usepackage{bbold}
\usepackage{MnSymbol,wasysym}
%%% ToC (table of contents) APPEARANCE
\usepackage[nottoc,notlof,notlot]{tocbibind} % Put the bibliography in the ToC
\usepackage[titles,subfigure]{tocloft} % Alter the style of the Table of Contents
\renewcommand{\cftsecfont}{\rmfamily\mdseries\upshape}
\renewcommand{\cftsecpagefont}{\rmfamily\mdseries\upshape} % No bold!

%%% END Article customizations

%%% The "real" document content comes below...

\title{HW10}
\author{Wei Ye\footnote{I worked on my assignment sololy. Email: wye22@fordham.edu}  	\\
	ECON 5700}
\date{Due on August 25, 2021.}


\begin{document}
	\maketitle
	\section{Question 1}
	\textbf{Solution:}
	
	\begin{itemize}
		\item $2>0$
		\item $2-1=1>0$
		\item 
		$$-(-1)(-2+b)+2(4-b^2)-(-1)(-2+b)=-2b^2+2b+4$$		
		\end{itemize}
			If $b\in (-1,2)$, it's positive definite. if $b\in [-1,2]$, it's positive semidefinite.
	
	
\section{Question 2}
\textbf{Solution:}

$$\begin{bmatrix}
	4&12&-16\\
	12&37&-43\\
	-16&-43&98
\end{bmatrix}\xrightarrow[R_2-3R_1]{R_3-(-4)R_1}\begin{bmatrix}
4&12&-16\\
0&1&5\\
0&5&34
\end{bmatrix}\xrightarrow{R_3-5R_2}\begin{bmatrix}
4&12&-16\\
0&1&5\\
0&0&9
\end{bmatrix}=U$$

$$L=\begin{bmatrix}
	1&0&0\\
	3&1&0\\
	-4&5&1
\end{bmatrix}$$
Since from the conclusion we deduced in class: $S=L^{T}=\begin{bmatrix}
	1&3&-4\\
	0&1&5\\
	0&0&1
\end{bmatrix}$. $\smiley$

\section{Question 3}
\textbf{Solution:}

\begin{align*}
	A&=LDL^{T}\\
	&=LddL^{T}\\
	&=R^{T}R\\
	&=\begin{bmatrix}
		1&0&0\\
		3&1&0\\
		-4&5&1
	\end{bmatrix}\begin{bmatrix}
	2&0&0\\
	0&1&0\\
	0&0&3
\end{bmatrix}\begin{bmatrix}
2&0&0\\
0&1&0\\
0&0&3
\end{bmatrix}\begin{bmatrix}
1&3&-4\\
0&1&5\\
0&0&1
\end{bmatrix}\\
&=\begin{bmatrix}
	4&0&0\\
	0&1&9\\
	0&0&9
\end{bmatrix}
\end{align*}



\section{Question 4}
\textbf{Solution:}

$$f(x,y)=x^2+3xy+y^2-x+3x$$
$$s.t.\ x+y-1=0$$
$$\mathcal{L}=x^2+3xy+y^2+2x+\lambda(x+y-1)$$
\begin{equation}
	\frac{\partial \mathcal{L}}{\partial x}=2x+3y+2+\lambda=0
\end{equation}
\begin{equation}
	\frac{\partial \mathcal{L}}{\partial y}=3x+2y+\lambda=0
\end{equation}
\begin{equation}
	\frac{\partial \mathcal{L}}{\partial \lambda}=x+y-1=0
\end{equation}

From these equations, we can deduce that $x^*=\frac{3}{2}, y^*=-\frac{1}{2}$. 
	
	
	
	
	
	
	
	
	
	
	
	
	
	
	
	
	
	
	
	
	
	
	
	
	
	
	
	
	
	
	
	
	
	
	
	
	
	
	
	
	
	
	
	
	
	
	
	
	
	
	
	
	
	
	
	
	
	
	
	
	
	
	
	
	
	
	
	
	
	
	
	
	
	
	
	
	
	
	
	
	
	
	
	
	
	
	
	
	
	
	
	
	
	
	
	
	
	
	
	
	
\end{document}