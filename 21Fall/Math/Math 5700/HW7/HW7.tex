%EX TS-program = pdflatex
% !TEX encoding = UTF-8 Unicode

% This is a simple template for a LaTeX document using the "article" class.
% See "book", "report", "letter" for other types of document.

\documentclass[11pt]{article} % use larger type; default would be 10pt

\usepackage[utf8]{inputenc} % set input encoding (not needed with XeLaTeX)

%%% Examples of Article customizations
% These packages are optional, depending whether you want the features they provide.
% See the LaTeX Companion or other references for full information.
\usepackage{amsmath}
\makeatletter
\renewcommand*\env@matrix[1][*\c@MaxMatrixCols c]{%
	\hskip -\arraycolsep
	\let\@ifnextchar\new@ifnextchar
	\array{#1}}
\makeatother
%%% PAGE DIMENSIONS
\usepackage{geometry} % to change the page dimensions
\usepackage{marvosym}
\geometry{a4paper} % or letterpaper (US) or a5paper or....
% \geometry{margin=2in} % for example, change the margins to 2 inches all round
% \geometry{landscape} % set up the page for landscape
%   read geometry.pdf for detailed page layout information

\usepackage{graphicx} % support the \includegraphics command and options
% \usepackage[parfill]{parskip} % Activate to begin paragraphs with an empty line rather than an indent
\usepackage{amssymb}
%%% PACKAGES
\usepackage{booktabs} % for much better looking tables
\usepackage{array} % for better arrays (eg matrices) in maths
\usepackage{paralist} % very flexible & customisable lists (eg. enumerate/itemize, etc.)
\usepackage{verbatim} % adds environment for commenting out blocks of text & for better verbatim
\usepackage{subfig} % make it possible to include more than one captioned figure/table in a single float
% These packages are all incorporated in the memoir class to one degree or another...
\usepackage{pgfplots}
%%% HEADERS & FOOTERS
\usepackage{fancyhdr} % This should be set AFTER setting up the page geometry
\pagestyle{fancy} % options: empty , plain , fancy
\renewcommand{\headrulewidth}{0pt} % customise the layout...
\lhead{}\chead{}\rhead{}
\lfoot{}\cfoot{\thepage}\rfoot{}

%%% SECTION TITLE APPEARANCE
\usepackage{sectsty}
\allsectionsfont{\sffamily\mdseries\upshape} % (See the fntguide.pdf for font help)
% (This matches ConTeXt defaults)
\usepackage[thinc]{esdiff}
\usepackage{bbold}
\usepackage{MnSymbol,wasysym}
%%% ToC (table of contents) APPEARANCE
\usepackage[nottoc,notlof,notlot]{tocbibind} % Put the bibliography in the ToC
\usepackage[titles,subfigure]{tocloft} % Alter the style of the Table of Contents
\renewcommand{\cftsecfont}{\rmfamily\mdseries\upshape}
\renewcommand{\cftsecpagefont}{\rmfamily\mdseries\upshape} % No bold!

%%% END Article customizations

%%% The "real" document content comes below...

\title{HW7}
\author{Wei Ye\footnote{I worked on my assignment sololy. Email: wye22@fordham.edu}  	\\
	ECON 5700}
\date{Due on August 22, 2020.}


\begin{document}
	\maketitle
	\section{Question 1}
	\textbf{Solution:}

\begin{enumerate}
	\item 
	
	let 
$\vec{u}= \begin{pmatrix}
	x_1\\
	y_1\\
	z_1
\end{pmatrix}$, $\vec{v}=
\begin{pmatrix}
	x_2\\
	y_2\\
	z_2
\end{pmatrix}$, and $\vec{m}= \begin{pmatrix}
x_3\\
y_3\\
z_3
\end{pmatrix}$.
\begin{align*}
	\mathcal{T}(\vec{u}+\vec{v}+\vec{m})&=\mathcal{T}(\begin{pmatrix}
			x_1\\
		y_1\\
		z_1
	\end{pmatrix}+
\begin{pmatrix}
x_2\\
y_2\\
z_2
\end{pmatrix}+
\begin{pmatrix}
x_3\\
y_3\\
z_3
\end{pmatrix})\\
&=\mathcal{T}(\begin{pmatrix}
	x_1+x_2+x_3\\
	y_1+y_2+y_3\\
	z_1+z_2+z_3
\end{pmatrix})\\
&=\begin{pmatrix}
	x_1-y_1+z_1\\
	2x_1+y_1-3z_1\\
	z_1
\end{pmatrix}+
\begin{pmatrix}
	x_2-y_2+z_2\\
	2x_2+y_2-3z_2\\
	z_2
\end{pmatrix}+
\begin{pmatrix}
	x_3-y_3+z_3\\
	2x_3+y_3-3z_3\\
	z_3
\end{pmatrix}\\
&=\mathcal{T}(\vec{u})+\mathcal{T}(\vec{v})+\mathcal{T}(\vec{m})
\end{align*}

Now, we prove under scalar mulplication, TL still exists. Sill let $\vec{v}=\begin{pmatrix}
	x\\
	y\\
	z
\end{pmatrix}$. 

\begin{align*}
	\mathcal{T}(c\vec{v})&=\mathcal{T}(\begin{pmatrix}
		cx\\
		cy\\
		cz
	\end{pmatrix})\\
&=\begin{pmatrix}
	c(x-y+z)\\
	c(2x+y-3z)\\
	cz
\end{pmatrix}\\
&=c\mathcal{T}(\vec{v})
\end{align*}

Thus, we can prove this is linear transformation.

\item
Let $\vec{u}=\begin{pmatrix}
	x_1\\
	y_1
\end{pmatrix}$, and $\vec{v}=\begin{pmatrix}
x_2\\
y_2
\end{pmatrix}$.
\begin{align*}
	\mathcal{T}(\vec{u}+\vec{v})&=\mathcal{T}(\begin{pmatrix}
		x_1+x_2\\
		y_1+y_2
	\end{pmatrix})\\
&=\begin{pmatrix}
	-y_1-y_2\\
	x_1+x_2+2y_1+2y_2\\
	3x_1+3x_2-4y_1-4y_2
\end{pmatrix}\\
&=\begin{pmatrix}
	-y_1\\
	x_1+2y_1\\
	3x_1-4y_1
\end{pmatrix}+\begin{pmatrix}
-y_2\\
x_2+2y_2\\
3x_2-4y_2
\end{pmatrix}\\
&=\mathcal{T}(\vec{u})+\mathcal{T}(\vec{v})
\end{align*}
New, we prove even in scalar multiplication, LT still exists.
\begin{align*}
	\mathcal{T}(c\begin{pmatrix}
		x\\
		y
	\end{pmatrix})&=\begin{pmatrix}
	c\begin{pmatrix}
		-y\\
		x+2y\\
		3x-4y
	\end{pmatrix}
\end{pmatrix}\\
&=c\begin{pmatrix}
	-y\\
	x+2y\\
	3x-4y
\end{pmatrix}\\
&=c\mathcal{T}(\begin{pmatrix}
	x\\
	y
\end{pmatrix})
\end{align*}

All in one, we can prove LT exist in our case.$\smiley$
\end{enumerate}


\section{Question 2}
\textbf{Solution:}
Since $T_A$ is matrix transformation. 

$$T_A(u)= \begin{bmatrix}
	2&-1\\
	3&4
\end{bmatrix}\begin{bmatrix}
1\\2
\end{bmatrix}=\begin{bmatrix}
0\\11
\end{bmatrix}$$
$$T_A(v)=\begin{bmatrix}
	2&-1\\2&4
\end{bmatrix}\begin{bmatrix}
3\\2
\end{bmatrix}=\begin{bmatrix}
4\\17
\end{bmatrix}$$.



\section{Question 3}
\textbf{Solution:}

The first and the second matrices are both not linear transformation. 

Because the first matrix does not exist under scalar mulplication. 
\begin{align*}
	\mathcal{T}(c\begin{bmatrix}
		x\\y
	\end{bmatrix})&=\begin{bmatrix}
	cy\\
	c^2x^2
\end{bmatrix}\\
&\neq c\mathcal{T}(\begin{bmatrix}
	x\\y
\end{bmatrix})
\end{align*}
For the second matrix, it doesn't exist due to the same reason. 
\begin{align*}
	\mathcal{T}(c\begin{bmatrix}
		x\\y
	\end{bmatrix})&=\begin{bmatrix}
	cxcy\\
	c(x+y)
\end{bmatrix}\\
&\neq c\mathcal{T}(\begin{bmatrix}
	x\\
	y
\end{bmatrix})
\end{align*}
Thus, they are not linear transformation. 

\section{Question 4}
\textbf{Solution:}

\begin{enumerate}
	\item By direct subsititution:
	\begin{align*}
		S^\circ T(a)&=S\begin{bmatrix}
			x_1+2x_2\\
			2x_2-x_3
		\end{bmatrix}\\
	&=\begin{bmatrix}
		x_1+2x_2-(2x_2-x_3)\\
		x_1+2x_2+2x_2-x_3\\
		-x_1-2x_2+2x_2-x_3
	\end{bmatrix}\\
&=\begin{bmatrix}
	x_1+x_3\\
	x_1+4x_2-x_3\\
	-x_1-x_3
\end{bmatrix}
	\end{align*}

\item By matrix multiplication:

$$S=y_1\begin{bmatrix}
	1\\
	1\\
	1
\end{bmatrix}+y_2\begin{bmatrix}
-1\\
1\\
1
\end{bmatrix}$$

$$ [S ]=\begin{bmatrix}
	1&-1\\
	1&1\\
	-1&1
\end{bmatrix}
$$
$$T=\begin{bmatrix}
	x_1+2x_2\\
	2x_2-x_3
\end{bmatrix}=x_1\begin{bmatrix}
1\\
0
\end{bmatrix}+x_2\begin{bmatrix}
2\\
2
\end{bmatrix}+x_3\begin{bmatrix}
0\\-1
\end{bmatrix}$$
$$[T]=\begin{bmatrix}
	1&2&0\\
	0&2&-1
\end{bmatrix}$$
Thus:

\begin{align*}
	[S][T]&=\begin{bmatrix}
		1&-1\\
		1&1\\
		-1&1
	\end{bmatrix}\begin{bmatrix}
	1&2&0\\
	0&2&-1
\end{bmatrix}\\
&=\begin{bmatrix}
	1&0&1\\
	1&4&-1\\
	-1&0&-1
\end{bmatrix}
\end{align*}
\end{enumerate}


\section{Question 5}

\textbf{Solution:}

\begin{align*}
	|A-\lambda I|&=\begin{bmatrix}
		-\lambda & 1&1\\
		1&-\lambda&1\\
		1&1&-\lambda
	\end{bmatrix}\\
&=-\lambda^3+3\lambda+2=0
\end{align*}
Thus, $\lambda_1=-1$, and $\lambda_2=2$. 
\begin{itemize}
	\item When $\lambda=-1$:
	\begin{align*}
		\begin{bmatrix}[ccc|c]
			1&1&1&0\\
			1&1&1&0\\
			1&1&1&0
		\end{bmatrix}&=\begin{bmatrix}[ccc|c]
		1&1&1&0\\
		0&0&0&0\\
		0&0&0&0
	\end{bmatrix}\\
&=x_1+x_2+x_3=0
	\end{align*}
Let $x_1=t$, $x_2=m$, and $x_3=-t-m$. 
$$\begin{bmatrix}
	t\\
	m\\
	-t-m
\end{bmatrix}=t\begin{bmatrix}
1\\0\\-1
\end{bmatrix}+m\begin{bmatrix}
0\\1\\-1
\end{bmatrix}$$
Thus, the eigenvectco associate with the eigenvalue of -1 is $\begin{pmatrix}
	1&0\\
	0&1\\
	-1&-1
\end{pmatrix}$.
\item When $\lambda=2:$
$$\begin{pmatrix}[ccc|c]
	-2&1&1&0\\
	1&-2&1&0\\
	1&1&-2&0
\end{pmatrix}=\begin{pmatrix}[ccc|c]
1&0&0&0\\
0&1&0&0\\
0&0&1&0
\end{pmatrix}$$
Thus, in the case the eigenvector is $\begin{pmatrix}
	1&0&0\\
	0&1&0\\
	0&0&1
\end{pmatrix}$

\end{itemize}






































































\end{document}