%EX TS-program = pdflatex
% !TEX encoding = UTF-8 Unicode

% This is a simple template for a LaTeX document using the "article" class.
% See "book", "report", "letter" for other types of document.

\documentclass[11pt]{article} % use larger type; default would be 10pt

\usepackage[utf8]{inputenc} % set input encoding (not needed with XeLaTeX)

%%% Examples of Article customizations
% These packages are optional, depending whether you want the features they provide.
% See the LaTeX Companion or other references for full information.
\usepackage{amsmath}
\makeatletter
\renewcommand*\env@matrix[1][*\c@MaxMatrixCols c]{%
	\hskip -\arraycolsep
	\let\@ifnextchar\new@ifnextchar
	\array{#1}}
\makeatother

\newcommand{\norm}[1]{\left\lVert#1\right\rVert}
%%% PAGE DIMENSIONS
\usepackage{geometry} % to change the page dimensions
\usepackage{listings}
\usepackage[dvipsnames]{xcolor}
\usepackage{marvosym}
\geometry{a4paper} % or letterpaper (US) or a5paper or....
% \geometry{margin=2in} % for example, change the margins to 2 inches all round
% \geometry{landscape} % set up the page for landscape
%   read geometry.pdf for detailed page layout information

\usepackage{graphicx} % support the \includegraphics command and options
% \usepackage[parfill]{parskip} % Activate to begin paragraphs with an empty line rather than an indent
\usepackage{amssymb}
%%% PACKAGES
\usepackage{booktabs} % for much better looking tables
\usepackage{array} % for better arrays (eg matrices) in maths
\usepackage{paralist} % very flexible & customisable lists (eg. enumerate/itemize, etc.)
\usepackage{verbatim} % adds environment for commenting out blocks of text & for better verbatim
\usepackage{subfig} % make it possible to include more than one captioned figure/table in a single float
% These packages are all incorporated in the memoir class to one degree or another...
\usepackage{pgfplots}
%%% HEADERS & FOOTERS
\usepackage{fancyhdr} % This should be set AFTER setting up the page geometry
\pagestyle{fancy} % options: empty , plain , fancy
\renewcommand{\headrulewidth}{0pt} % customise the layout...
\lhead{}\chead{}\rhead{}
\lfoot{}\cfoot{\thepage}\rfoot{}

%%% SECTION TITLE APPEARANCE
\usepackage{sectsty}
\allsectionsfont{\sffamily\mdseries\upshape} % (See the fntguide.pdf for font help)
% (This matches ConTeXt defaults)
\usepackage[thinc]{esdiff}
\usepackage{bbold}
\usepackage{MnSymbol,wasysym}
%%% ToC (table of contents) APPEARANCE
\usepackage[nottoc,notlof,notlot]{tocbibind} % Put the bibliography in the ToC
\usepackage[titles,subfigure]{tocloft} % Alter the style of the Table of Contents
\renewcommand{\cftsecfont}{\rmfamily\mdseries\upshape}
\renewcommand{\cftsecpagefont}{\rmfamily\mdseries\upshape} % No bold!

%%% END Article customizations

%%% The "real" document content comes below...

\title{HW3}
\author{Wei Ye\footnote{ 1st year PhD student in Economics Department at Fordham University. Email: wye22@fordham.edu}
    \\ ECON 7910 Econometrics}
\date{Due on Oct 7, 2021}

\begin{document}
	\maketitle

\section{Question -- 4.1}
\textbf{Solution:}
\begin{enumerate}
    \item Because $\log(wage)=\beta_0+\beta_1married+\beta_2edu+z\gamma+u$ as our equation, rearrange this to obtain:
          \begin{align*}
              wage&=\exp(\beta_0+\beta_1married+\beta_2edu+z\gamma+u)\\
                  &= \exp(u)\exp(\beta_0+\beta_1married+\beta_2edu+z\gamma)
          \end{align*}
          The reason why we plug $\exp(u)$ out the above equation is that $u$ is independent with our covariate variabes, i.e.,
          $E(u|married,edu,\gamma)=E(u)=constant$. Thus, we can derive an equation:
          \begin{align*}
              \hat{\theta}&=100(\frac{\exp(u)\exp(\beta_0+\beta_1+\beta_2edu+z\gamma)-\exp(u)\exp(\beta_0+\beta_2edu+z\gamma)}{\exp{u}\exp(\beta_0+\beta_2edu+z\gamma)})\\
                          &= 100(\exp(\hat{\beta_1})-1)
          \end{align*}
    \item Since from (1), we already have $\theta_1=100(\exp(\beta_1)-1)$. By delta method:
        \begin{equation*}
            \sqrt{N}(\theta_1-100(\exp(\beta_1)-1))\xrightarrow{a} \mathcal{N}(0,(100\exp(\beta_1))avar(\beta_1)(100\exp(\beta_1))')
        \end{equation*}
        Thus, the standard error of $\hat{\theta_1}$ is $100\exp(\hat{\beta_1})se(\hat{\beta_1})$

        \textcolor{red}{One thing is unclear is how to assume $N=1$, because, if $N\neq 1$, then the asymptotic standard error will be with N in its final result.}
    \item While estimating the effect of change educ to wage, we need to fix other variables. 
      \begin{align*}
          \theta_2&=\frac{\beta_2\exp(edu_1)-\exp(\beta_2edu_0)}{\exp(\beta_2edu_0)}\\
                  &= \exp(\beta_2(edu_1-edu_0))-1
      \end{align*}
      Thus, with same logistics with (1): $\hat{\theta_2}=100\exp(\hat{\beta_2}(edu_1-edu_0))$

      About estimating asymptotic se of $\hat(\theta_2)$, the step is the same, but for the convenience of computation, we assume $edu_1-edu_0=\Delta edu$.
      \begin{equation*}
          \sqrt{N}(\hat{\theta_2}-100\exp(\hat{\beta_2}\Delta edu))\xrightarrow{a}\mathcal{N}(0,(100\Delta edu \exp(\hat{\beta_2}\Delta edu))Avar(\hat{\beta_2})(100\Delta edu \exp(\hat{\beta_2}\Delta edu))')
      \end{equation*}
      Thus, the se of $\hat{\theta_1}$ is $100\Delta edu \exp(\hat{\beta_2}\Delta edu)\cdot se(\hat{\beta_2})$.
    \item For the first regression, see tha table \ref{Question1}: The $\hat{\theta_1}=0.065$, and $se(\hat{\theta_1})=0.006250$.
      
      The $\theta_2=0.0654$, and $se=0.063$ (\textcolor{Cyan}{I actually don't know how to write R codes for $\Delta educ=4$, because there is only code example on google like $\Delta=1$.})
      For specific R codes, as below:
      \begin{lstlisting}[language=R]
        library(tidyverse)
        library(stargazer)
        library(margins)
        #Question1-4.1(d)
        wage_1<- read_csv('nls80.csv')
        head(wage_1,n=5)
        lwage_reg<-lm(lwage~married+educ+(exper+tenure+south+urban+black),data=wage_1)
        summary(lwage_reg)
        stargazer(lwage_reg)
        #For estimating the margin effect of educ with 4, need to dig out.
        summary(margins(lwage_reg,variables='educ'))
      \end{lstlisting}
\end{enumerate}


\section{Question -- 4.2}
\textbf{Solution:}
\begin{enumerate}
    \item  \begin{align*}
        E(\hat{\beta}|X)&=E((X'X)^{-1}X'Y)\\
                        &=E(\beta+(X'X)^{-1}X'u|X)\\
                        &=\beta+(X'X)^{-1}X'E(u|X)\\
                        &=\beta+0\\
                        &=\beta
    \end{align*}
    \item \begin{align*}
        Var(\hat{\beta}^2|X)&=E(((X'X)^{-1}X'Y)^2|X)-\beta^2\\
                            &=\beta^2-\beta^2+E(((X'X)^{-1}X'U)^2|X)\\
                            &=(X'X)^{-1}X'X(X'X)^{-1}E(U^2|X)\\
                            &=(X'X)^{-1}\sigma^2
    \end{align*}
\end{enumerate}

\section{Question -- 4.3}
\textbf{Solution:}
\begin{enumerate}
    \item Yes, the reason is as below:
         \begin{align*}
             E(u^2|x)&=Var(u|x)+E(u|x)^2\\
                     &=Var(u|x)+0\\
                     &=\sigma^2
         \end{align*}
    \item It's homoskedasticity, if it doesn't hold, there would some error terms in our regression models.
\end{enumerate}


\section{Question -- 4.5}
\textbf{Solution:}
\begin{enumerate}
    \item First, since $\hat{\beta}$ is the coefficient of regression regarding to $x$ and $z$, and it's consistent with $\beta$.
        We can derive $\sqrt{N}(\hat{\beta}-\beta)\xrightarrow{a}\mathcal{N}(0,Avar(\beta))$ by CLT. And also for $\sqrt{N}(\tilde{\beta}-\beta)\xrightarrow{a}\mathcal{N}(0,Avar(\beta))$. But since $\tilde{\beta}$ is only from the regression of y w/ respect to x, however, $\hat{\beta}$ is for the integration of $x$ and $z$.
        From mathematical estimation perspective, The CLT of latter is weakly higher than the former, which means the substraction is positively semidefinite.

        \textcolor{violet}{Feel free to deduct points, because I can't convince myself in some sense, either.}
    \item I think it's better to use. Because in two-dimension, we can't only measure partial effect, but measurement errors. But in one dimensional space, it can't.
    \item Why not in question 2.3? But in this question, it's better to not use, because there is an unobservable variable $z$, which viotates the assumption of uncorrelated covariates in question 2.3.
\end{enumerate}
























































\section{Appendix}
\begin{table}[!htbp] \centering 
    \caption{} 
    \label{Question1}
  \begin{tabular}{@{\extracolsep{5pt}}lc} 
  \\[-1.8ex]\hline 
  \hline \\[-1.8ex] 
   & \multicolumn{1}{c}{\textit{Dependent variable:}} \\ 
  \cline{2-2} 
  \\[-1.8ex] & lwage \\ 
  \hline \\[-1.8ex] 
   married & 0.199$^{***}$ \\ 
    & (0.039) \\ 
    & \\ 
   educ & 0.065$^{***}$ \\ 
    & (0.006) \\ 
    & \\ 
   exper & 0.014$^{***}$ \\ 
    & (0.003) \\ 
    & \\ 
   tenure & 0.012$^{***}$ \\ 
    & (0.002) \\ 
    & \\ 
   south & $-$0.091$^{***}$ \\ 
    & (0.026) \\ 
    & \\ 
   urban & 0.184$^{***}$ \\ 
    & (0.027) \\ 
    & \\ 
   black & $-$0.188$^{***}$ \\ 
    & (0.038) \\ 
    & \\ 
   Constant & 5.395$^{***}$ \\ 
    & (0.113) \\ 
    & \\ 
  \hline \\[-1.8ex] 
  Observations & 935 \\ 
  R$^{2}$ & 0.253 \\ 
  Adjusted R$^{2}$ & 0.247 \\ 
  Residual Std. Error & 0.365 (df = 927) \\ 
  F Statistic & 44.747$^{***}$ (df = 7; 927) \\ 
  \hline 
  \hline \\[-1.8ex] 
  \textit{Note:}  & \multicolumn{1}{r}{$^{*}$p$<$0.1; $^{**}$p$<$0.05; $^{***}$p$<$0.01} \\ 
  \end{tabular} 
  \end{table}



\end{document}