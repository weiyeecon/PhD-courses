%EX TS-program = pdflatex
% !TEX encoding = UTF-8 Unicode

% This is a simple template for a LaTeX document using the "article" class.
% See "book", "report", "letter" for other types of document.

\documentclass[11pt]{article} % use larger type; default would be 10pt

\usepackage[utf8]{inputenc} % set input encoding (not needed with XeLaTeX)

%%% Examples of Article customizations
% These packages are optional, depending whether you want the features they provide.
% See the LaTeX Companion or other references for full information.
\usepackage{amsmath}
\makeatletter
\renewcommand*\env@matrix[1][*\c@MaxMatrixCols c]{%
	\hskip -\arraycolsep
	\let\@ifnextchar\new@ifnextchar
	\array{#1}}
\makeatother

\newcommand{\norm}[1]{\left\lVert#1\right\rVert}
%%% PAGE DIMENSIONS
\usepackage{geometry} % to change the page dimensions
\usepackage{marvosym}
\geometry{a4paper} % or letterpaper (US) or a5paper or....
% \geometry{margin=2in} % for example, change the margins to 2 inches all round
% \geometry{landscape} % set up the page for landscape
%   read geometry.pdf for detailed page layout information

\usepackage{graphicx} % support the \includegraphics command and options
% \usepackage[parfill]{parskip} % Activate to begin paragraphs with an empty line rather than an indent
\usepackage{amssymb}
%%% PACKAGES
\usepackage{booktabs} % for much better looking tables
\usepackage{array} % for better arrays (eg matrices) in maths
\usepackage{paralist} % very flexible & customisable lists (eg. enumerate/itemize, etc.)
\usepackage{verbatim} % adds environment for commenting out blocks of text & for better verbatim
\usepackage{subfig} % make it possible to include more than one captioned figure/table in a single float
% These packages are all incorporated in the memoir class to one degree or another...
\usepackage{pgfplots}
%%% HEADERS & FOOTERS
\usepackage{fancyhdr} % This should be set AFTER setting up the page geometry
\pagestyle{fancy} % options: empty , plain , fancy
\renewcommand{\headrulewidth}{0pt} % customise the layout...
\lhead{}\chead{}\rhead{}
\lfoot{}\cfoot{\thepage}\rfoot{}

%%% SECTION TITLE APPEARANCE
\usepackage{sectsty}
\allsectionsfont{\sffamily\mdseries\upshape} % (See the fntguide.pdf for font help)
% (This matches ConTeXt defaults)
\usepackage[thinc]{esdiff}
\usepackage{bbold}
\usepackage{MnSymbol,wasysym}
%%% ToC (table of contents) APPEARANCE
\usepackage[nottoc,notlof,notlot]{tocbibind} % Put the bibliography in the ToC
\usepackage[titles,subfigure]{tocloft} % Alter the style of the Table of Contents
\renewcommand{\cftsecfont}{\rmfamily\mdseries\upshape}
\renewcommand{\cftsecpagefont}{\rmfamily\mdseries\upshape} % No bold!

%%% END Article customizations

%%% The "real" document content comes below...

\title{HW2}
\author{Wei Ye\footnote{ 1st year PhD student in Economics Department at Fordham University. Email: wye22@fordham.edu}
    \\ ECON 7910 Econometrics}
\date{Due on Sep 30, 2021}

\begin{document}
	\maketitle

\section{Question -- 3.1}
Prove Lemma 3.1

\textbf{Solution:}

Since $N^0x_N=x_N$, meanwhile, from the question, we know $x_N \xrightarrow{P} a$, where a is a constant. Thus $N^0x_N$ is bounded. 
From these, we can get the conclusion: $x_N=O_p(1)$.

\section{Question -- 3.3}
Prove under the assumption of lemma 3.4, we have $g(x_N)=O_p(1)$.

\textbf{Solution:}

Since from the contents of lemma, $x_N \xrightarrow{P} c$ and $g(x_N)\xrightarrow{P} g(c)$. Since
$g(c)$ is bounded, which means $g(x_N)$ is bounded as well. $N^0 g(x_N)$ is bounded $\longrightarrow$ $g(x_N)=O_p(1)$.

\section{Question -- 3.5}
\textbf{Solution:}

\begin{enumerate}
    \item The sample averge is: $\frac{\sigma^2}{N}$. Since $\sqrt{N}(\bar{y}_N-\mu) ~ \mathcal{N}(0,\sigma^2)$. Thus, 
    $Var(\sqrt{N}(\bar{y}_N-\mu))=\sigma^2$. 
    \item The asymptotic variance is $\sigma^2$.
    \item $AVar(\bar{y}_N)=\frac{\sigma^2}{N}$. $Var(\bar{y})=\frac{\sigma^2}{N}$ from question (a). They are equal in this case.
    \item The standard variance of $\bar{y}_N$ is $\frac{\sigma}{\sqrt{N}}$.
    \item First, get the asymptotic variance of $\bar{y}_N$, then make square root of this variance to get 
    asymptotic standard error.
\end{enumerate}


\section{Question -- 3.7}
\textbf{Solution:}

\begin{enumerate}
    \item From Slusky Theorem, $plim(\log(\hat{\theta}))=\log(plim(
        \hat{\theta}))$. Thus, $\hat{\gamma}$ is a consistent estimator of $\gamma$.
    \item First, assume the asymptotic variance of $\sqrt{N}(\hat{\theta}-\theta)$ as 
        $AVar(\sqrt{N}(\hat{\theta}-\theta))$. And, $\hat{\gamma}=
        \log(\hat{\theta})$ Thus, $AVar(\sqrt{N}(\hat{\gamma}-\gamma))=
        AVar(\sqrt{N}(\log(\hat{\theta})-\log(\theta)))$. By Delta Method:
        \begin{equation*}
            Avar(\sqrt{N}(\hat{\gamma}-\gamma))=\frac{1}{\theta}Avar(\sqrt{N}(\hat{\theta}-\theta))\frac{1}{\theta}=(\frac{1}{\theta})^2 Avar(\sqrt{N}(\hat{\theta}-\theta))
        \end{equation*}
    \item  $\hat{\theta}=4$ and $se(\hat{\theta})=2$. The trick is assuming $Avar(\sqrt{N}(\hat{\theta}-\theta))=\sigma^2$,
        Then, $se(\hat{\theta})=(\frac{\sigma^2}{N})^{\frac{1}{2}} \longrightarrow \frac{\sigma^2}{N}=4$. Thus, $\sigma^2=4N$. 
        Go on our calculation $se(\hat{\gamma})=(\frac{4N\frac{1}{\theta^2}}{N\theta^2})=\frac{2}{\theta}$.
        Finally, plug $\theta=4$ into $se(\hat{\gamma})=\frac{2}{4}=\frac{1}{2}$.
    \item For null hypothesis $H_0:\theta=1$, if we have been given some information from part(c), then $|t|=|\frac{1-4}{2}|=|\frac{-3}{2}|$. We need to compare the t value from computation with t-criterion table to determine in which confidence interval we reject our null hypothesis.
    \item This question is a transformation one from one variable to another linear or non-linear variable, aka, Wald statistic. $H_0: \gamma=0$. $t=\frac{\log(4)-log(1)}{se(\hat{\gamma})}=\frac{\log(4)}{\frac{1}{2}}=2\log(4)$. In the final step, we need to repeat the step we did in the last question, to compare t value with the value in the t-table given some confidence interval. 
\end{enumerate}




\section{Question -- 3.8}
\textbf{Solution:}

\begin{enumerate}
    \item Since $\hat{\theta}$ is the asymptotic normal estimation of $\theta$, and $\hat{\gamma}=\frac{\hat{\theta_1}}{\hat{\theta_2}}$ is the estimation of 
    $\gamma=\frac{\theta_1}{\theta_2}$. By the continuous function of $\theta$. 
    \begin{equation*}
        plim \hat{\gamma}=plim \frac{\hat{\theta_1}}{\hat{\theta_2}}=\frac{\theta_1}{\theta_2}=\gamma
    \end{equation*}
    \item By delta method, 
    \begin{equation*}
        AVar(\hat{\gamma})=(\frac{1}{\theta_2},-\frac{\theta_1}{\theta_2^2})Avar(\hat{\theta})(\frac{1}{\theta_2},-\frac{\theta_1}{\theta_2^2})'
    \end{equation*}
    \item Using the result of (b), we can derive:
    \begin{equation*}
        se(\hat{\gamma})=(\frac{Avar(\hat{\gamma})}{N})^{\frac{1}{2}}=\sqrt{71.2}\approx 8.438
    \end{equation*}
    Note: 1 sample size means $N=1$.
\end{enumerate}




























































\end{document}