%EX TS-program = pdflatex
% !TEX encoding = UTF-8 Unicode

% This is a simple template for a LaTeX document using the "article" class.
% See "book", "report", "letter" for other types of document.

\documentclass[11pt]{article} % use larger type; default would be 10pt
\usepackage[utf8]{inputenc} % set input encoding (not needed with XeLaTeX)

%%% Examples of Article customizations
% These packages are optional, depending whether you want the features they provide.
% See the LaTeX Companion or other references for full information.
\usepackage{amsmath}
\makeatletter
\renewcommand*\env@matrix[1][*\c@MaxMatrixCols c]{%
	\hskip -\arraycolsep
	\let\@ifnextchar\new@ifnextchar
	\array{#1}}
\makeatother

\newcommand{\norm}[1]{\left\lVert#1\right\rVert}
%%% PAGE DIMENSIONS
\usepackage{geometry} % to change the page dimensions
\usepackage{listings}
\usepackage[dvipsnames]{xcolor}
\usepackage{marvosym}
\geometry{a4paper} % or letterpaper (US) or a5paper or....
% \geometry{margin=2in} % for example, change the margins to 2 inches all round
% \geometry{landscape} % set up the page for landscape
%   read geometry.pdf for detailed page layout information

\usepackage{graphicx} % support the \includegraphics command and options
% \usepackage[parfill]{parskip} % Activate to begin paragraphs with an empty line rather than an indent
\usepackage{amssymb}
\usepackage{mathrsfs}
%%% PACKAGES
\usepackage{booktabs} % for much better looking tables
\usepackage{array} % for better arrays (eg matrices) in maths
\usepackage{paralist} % very flexible & customisable lists (eg. enumerate/itemize, etc.)
\usepackage{verbatim} % adds environment for commenting out blocks of text & for better verbatim
\usepackage{subfig} % make it possible to include more than one captioned figure/table in a single float
% These packages are all incorporated in the memoir class to one degree or another...
\usepackage{pgfplots}
%%% HEADERS & FOOTERS
\usepackage{fancyhdr} % This should be set AFTER setting up the page geometry
\pagestyle{fancy} % options: empty , plain , fancy
\renewcommand{\headrulewidth}{0pt} % customise the layout...
\lhead{}\chead{}\rhead{}
\lfoot{}\cfoot{\thepage}\rfoot{}

%%% SECTION TITLE APPEARANCE
\usepackage{sectsty}
\allsectionsfont{\sffamily\mdseries\upshape} % (See the fntguide.pdf for font help)
% (This matches ConTeXt defaults)
\usepackage[thinc]{esdiff}
\usepackage{bbold}
\usepackage{MnSymbol,wasysym}
%%% ToC (table of contents) APPEARANCE
\usepackage[nottoc,notlof,notlot]{tocbibind} % Put the bibliography in the ToC
\usepackage[titles,subfigure]{tocloft} % Alter the style of the Table of Contents
\renewcommand{\cftsecfont}{\rmfamily\mdseries\upshape}
\renewcommand{\cftsecpagefont}{\rmfamily\mdseries\upshape} % No bold!

%%% END Article customizations

%%% The "real" document content comes below...

\title{Homework 2}
\author{Wei Ye\footnote{ 1st year PhD student in Economics Department at Fordham University. Email: wye22@fordham.edu}
    \\ ECON 7010- Microeconomics II}
\date{Due on Feb 2, 2022}
\begin{document}
\maketitle

\section{Qeustion 1 -- 2.E.1}
\textbf{Solution:}

We apply $\mathcal{P}\cdot \mathcal{X}$: $\sum_i^3p_i*x_i=\frac{w(\beta p_1+p_2+p_3)}{p_1+p_2+p_3}$

\begin{itemize}
    \item When $\beta=1$, $\sum_i^3p_i*x_i=w$, it satisfies  Walras' law. $x_1(\alpha p,\alpha w)=\frac{\alpha p_2}{\alpha (p_1+p_2+p_3)}\frac{\alpha w}{\alpha p_1}=x_1(p,w)$. Using the same method to derive other two, we get the same result, thus, it satisfies homogeneity of degree zero.
    \item When $\beta\in (0,1)$, $\sum_i^3p_i*x_i<w$, it violates Walras' law. For $x_1$ and $x_2$, they are same with previous, so we only watch for $x_3$, $x_3(\alpha p,\alpha w)=\frac{\alpha \beta p_1}{\alpha (p_1+p_2+p_3)}\frac{\alpha w}{\alpha p_3}=x_3(p,w)$, therefore, it satisfies homogeneity of degree zero.
\end{itemize}

\section{Question 2 -- 2.E.7}
\textbf{Solution:}

By Walras' law, $p_1x_1(p,w)+p_2x_2(p,w)=w$, plugging $x_1=\frac{\alpha w}{p_1}$ into Walras' law equation. We can obtain:
$p_1\cdot \frac{\alpha w}{p_1}+p_2\cdot x_2(p,w)=w$, then deriving this equation, $x_2(p,w)=\frac{(1-\alpha)w}{p_2}$

To test homegeneity of degree zero:
$$x_1(\alpha p,\alpha w)=\frac{\alpha^2w}{\alpha p_1}=x_1(p,w)$$
\begin{equation*}
    x_2(\alpha p,\alpha w)=\frac{\alpha(1-\alpha)w}{\alpha p_2}=x_2(p,w)
\end{equation*}

Thus, her demand function satisfies homogeneity of degree zero.




\section{Question 3 -- 2.F.3}
\textbf{Solution:}

\begin{enumerate}[(a)]
    \item 
        This question is a little tricky, now if we assume her behavior is consistent, i.e., satisfing WARP, it means 
        $p_1x_1'+p_2x_2' \leq p_1x_1+p_2x_2$ and $p_1'x_1+p_2'x_2> p_1'x_1'+p_2'x_2'$, however, if it violates WARP, we need to make induction condition have converse direction, 
        which means $p_1'x_1+p_2'x_2 \leq p_1'x_1'+p_2'x_2'$
        
        Combine:
        \begin{equation}
            p_1x_1'+p_2x_2' \leq p_1x_1+p_2x_2 
        \end{equation}
        \begin{equation}
            p_1'x_1+p_2'x_2 \leq p_1'x_1'+p_2'x_2'
        \end{equation}

        Plug $p_1=100$, $p_1'=120$, $p_2=100$, $p_2'=80$, $x_1=100$, $x_1'=120$, $x_2=100$ into the above two equations.
        \begin{equation*}
            100*120+100x_2'\leq 100*100+100*100
        \end{equation*}
        \begin{equation*}
            100*100+80*100\leq 100*120+80x_2'
        \end{equation*}
Thus, $x_2'\in [75,80]$

        \item Since year 1's bundle is revealed preferred to the bundle in year 2.
            \begin{equation}
                p_1x_1'+p_2x_2'\leq  p_1x_1+p_2x_2
            \end{equation}
            \begin{equation}
                p_1'x_1+p_2'x_2>p_1'x_1'+p_2'x_2'
            \end{equation}
        Plug values into these equations:
            \begin{equation*}
                100*120+100x_2'\leq 100*100+100*100
            \end{equation*}
            \begin{equation*}
                100*100+80*100>100*100+80x_2'
            \end{equation*}
    Therefore, $x_2' <75$.
        \item As year 2's consumption bundle is revealed preferred to year 2's:
            \begin{equation}
                p_1'x_1+p_2'x_2\leq p_1'x_1'+p_2'x_2'
            \end{equation}
            \begin{equation}
                p_1x_1'+p_2x_2'>p_1x_1+p_2x_2
            \end{equation}
            Plug values into these two equations:
            \begin{equation*}
                100*100+80*100\leq 100*120+80x_2'
            \end{equation*}
            \begin{equation*}
                100*120>100*100+100*100
            \end{equation*}
    Hence, $x_2'>80$
\end{enumerate}


\section{Question 4 -- 2.F.16}
\textbf{Solution:}

\begin{enumerate}[(a)]
    \item $$x_1(\alpha p, \alpha w)=\frac{\alpha p_2}{\alpha p_3}=\frac{p_2}{p_3}=x_1(p,w)$$
        \begin{equation*}
            x_2(\alpha p,\alpha w)=-\frac{\alpha p_1}{\alpha p_3}=-\frac{p_1}{p_3}=x_2(p,w)
        \end{equation*}
        \begin{equation*}
            x_3(\alpha p, \alpha w)=\frac{\alpha w}{\alpha p_3}=\frac{w}{p_3}=x_3(p,w)
        \end{equation*}
        Thus, $x(p,w)$ satisfies homogeneity of degree zero.
        Now, let's  prove Walras' law:
        \begin{align*}
            \sum_i^3 p_ix_i&= p_1\cdot \frac{p_2}{p_3}+p_2\cdot \frac{-p_1}{p_3}+p_3\cdot \frac{2}{p_3}\\
                            &= \frac{p_1p_2-p_1p_2+wp_3}{p_3}\\
                            &=w
        \end{align*}
        It satisfies Walras' law.
    \item If x satisfies WARP, it should $px'\leq w$,  $p'x>w$,  we prove by contradiction:
        Let $p=(1,2,1)$, w=1, thus, $x=(2,-1,1)$, let $p'=(1,1,1)$, $w'=2$,thus, $x'=(1,-1,2)$.
        \begin{equation*}
            p*x'=2>w
        \end{equation*}
        \begin{equation*}
            p'*x=2=w'
        \end{equation*}
        Therefore, it violates WARP.
\end{enumerate}

\section{Question 5 -- 2.F.17}
\textbf{Solution:}

\begin{enumerate}[(a)]
    \item \begin{align*}
        x_k(\alpha x, \alpha w)&=\frac{\alpha w}{\alpha \sum_{l=1}^L p_l}\\
                                &=\frac{w}{\sum_{l=1}^L p_l}\\
                                &=x_k(x,w)
    \end{align*}
    Thus, it's homogeneous of degree zero.
    \item \begin{align*}
        \sum_k^Lx_k(p,w)p_k&=\sum_k^Lp_k\cdot \frac{w}{\sum_{l=1}^L p_l}\\
                            &= \frac{w}{\sum_{l=1}^L p_l}\sum_k^Lp_k\\
                            &=\frac{w}{\sum_{l=1}^L p_l}\cdot \sum_{l=1}^L p_l\\
                            &=w
    \end{align*}
    Yes, it satisfies Walras' law.
    \item Suppose $px_k(p',w')\leq w$ and $p'x_k(p,w)\leq w'$. From the first inequality, $\sum_k^L p_k \frac{w'}{\sum_{l=1}^L p_l}\leq w$, which means $\frac{w'}{\sum_{l=1}^L p_l'}\leq \frac{w}{\sum_{l=1}^L p_l}$,
    from the second inequality, $\sum_k^Lp_k'\frac{w}{\sum_{l=1}^L p_l}\leq w'$, thus, $\frac{w'}{\sum_{l=1}^L p_l'}\geq \frac{w}{\sum_{l=1}^L p_l}$,
    Hence, $x_k(p,w)=x_k(p',w')$, it satisfies WARP.  
\end{enumerate}






















\end{document}