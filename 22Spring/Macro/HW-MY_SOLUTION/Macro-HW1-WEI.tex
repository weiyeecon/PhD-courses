%EX TS-program = pdflatex
% !TEX encoding = UTF-8 Unicode

% This is a simple template for a LaTeX document using the "article" class.
% See "book", "report", "letter" for other types of document.

\documentclass[11pt]{article} % use larger type; default would be 10pt
\usepackage[utf8]{inputenc} % set input encoding (not needed with XeLaTeX)

%%% Examples of Article customizations
% These packages are optional, depending whether you want the features they provide.
% See the LaTeX Companion or other references for full information.
\usepackage{amsmath}
\makeatletter
\renewcommand*\env@matrix[1][*\c@MaxMatrixCols c]{%
	\hskip -\arraycolsep
	\let\@ifnextchar\new@ifnextchar
	\array{#1}}
\makeatother

\newcommand{\norm}[1]{\left\lVert#1\right\rVert}
%%% PAGE DIMENSIONS
\usepackage{geometry} % to change the page dimensions
\usepackage{listings}
\usepackage{hyperref}
\usepackage[dvipsnames]{xcolor}
\usepackage{marvosym}
\geometry{a4paper} % or letterpaper (US) or a5paper or....
% \geometry{margin=2in} % for example, change the margins to 2 inches all round
% \geometry{landscape} % set up the page for landscape
%   read geometry.pdf for detailed page layout information

\usepackage{graphicx} % support the \includegraphics command and options
% \usepackage[parfill]{parskip} % Activate to begin paragraphs with an empty line rather than an indent
\usepackage{amssymb}
\usepackage{mathrsfs}
%%% PACKAGES
\usepackage{booktabs} % for much better looking tables
\usepackage{array} % for better arrays (eg matrices) in maths
\usepackage{paralist} % very flexible & customisable lists (eg. enumerate/itemize, etc.)
\usepackage{verbatim} % adds environment for commenting out blocks of text & for better verbatim
\usepackage{subfig} % make it possible to include more than one captioned figure/table in a single float
% These packages are all incorporated in the memoir class to one degree or another...
\usepackage{pgfplots}
%%% HEADERS & FOOTERS
\usepackage{fancyhdr} % This should be set AFTER setting up the page geometry
\pagestyle{fancy} % options: empty , plain , fancy
\renewcommand{\headrulewidth}{0pt} % customise the layout...
\lhead{}\chead{}\rhead{}
\lfoot{}\cfoot{\thepage}\rfoot{}

%%% SECTION TITLE APPEARANCE
\usepackage{sectsty}
\allsectionsfont{\sffamily\mdseries\upshape} % (See the fntguide.pdf for font help)
% (This matches ConTeXt defaults)
\usepackage[thinc]{esdiff}
\usepackage{bbold}
\usepackage{MnSymbol,wasysym}
%%% ToC (table of contents) APPEARANCE
\usepackage[nottoc,notlof,notlot]{tocbibind} % Put the bibliography in the ToC
\usepackage[titles,subfigure]{tocloft} % Alter the style of the Table of Contents
\renewcommand{\cftsecfont}{\rmfamily\mdseries\upshape}
\renewcommand{\cftsecpagefont}{\rmfamily\mdseries\upshape} % No bold!

%%% END Article customizations

%%% The "real" document content comes below...

\title{Homework 1}
\author{Wei Ye\footnote{ 1st year PhD student in Economics Department at Fordham University. Email: wye22@fordham.edu}
    \\ ECON 7020- Macroeconomics II}
\date{Due on Feb 3, 2022}
\begin{document}
\maketitle

\section{Problems 1}
\textbf{Solution:}

\begin{enumerate}[a.]
	\item Since the transition function is $x'=g(x,u,z,\phi)$, it means next period of state variable is predetermined by this period of control variable u, state variable  x, one time shocks  z and $\phi$ (Impulse Response). 
	\item Set up Bellman Equation (Since there is only on time shock, we don't take them into value function):
		\begin{align*}
			V(x)&=\max_u\{r(x,u)+\beta E_t[V(x')]\}\\
				&= \max_u\{r(x,u)+\beta E_t[V(g(x,u,z,\phi))]\}
		\end{align*}
	\item \begin{itemize}
		\item FOC with respect to u:
		\begin{equation*}
			\frac{\partial r(x,u)}{\partial u}+\beta E_t\{V'(x')\frac{\partial g(x,u,z,\phi)}{\partial u}\}=0
		\end{equation*}
		\item Benveniste-Sheinkman Condition:
		\begin{equation*}
			v'(x)=\frac{\partial r(x,u)}{\partial x}+\beta E_t\{V'(x')\frac{\partial g(x,u,z,\phi)}{\partial x}\}
		\end{equation*}
	\end{itemize}
	\item In order to satisfy Banach Fixed Point Theorem, aka, Contraction Mapping Theorem, it should have monotonicity and discounting properties.
	At the same time, the process should be bounded on some value, and  is $\sigma$-measurable.
\end{enumerate}



\section{Problem 2}
\textbf{Solution:}

\begin{enumerate}[a.]
	\item Set up a bellman equation:
		\begin{align*}
			V(k,z)&=\max_c\{\ln(c)+\beta E_t[V(k',z')|z]\}\\
				&=\max_c\{\ln(c)+\beta  E_t[V(zk^\alpha-c,z')|z]\}
		\end{align*}
	\item The explicit form of Bellman Equation:
		\begin{equation*}
			V(k,z)=\max_c\{\ln(c)+\beta [V((p^Hz^H+(1-p^H)z^L)k^\alpha-c)]\}
		\end{equation*}
	\item WLOG, we guess and assume $V(k_{t+1})=\ln(k_{t+1})$. For iterations in Bellman Equation, We assume $V^0(k')=0$ like what we did in the lecture.
		In the meanwhile, WLOG, we denote $\bar{z}=p^Hz^H+(1-p^H)z^L$.

		The \textbf{first} iteration:
		\begin{equation*}
			V^1(k)=\max_{k'}\{\ln(\bar{z}k^\alpha-k')\}
		\end{equation*}
		In this case, it's obvious that when $k'=0$, we can get the maximum value $V^1(k)=\ln(\bar{z}k^\alpha)$. Plug this value into our second iteration:

		The \textbf{second} iteration:
		\begin{equation*}
			V^2(k)=\max_{k'}\{\ln(\bar{z}k^\alpha-k')+\beta(\ln(\bar{z})+\alpha\ln(k'))\}
		\end{equation*}
		Taking the first order derivative w.r.t $k'$, we can obtain $\frac{-1}{\bar{z}k^\alpha-k'}+\frac{\beta\alpha}{k'}=0$, thus, $k'=\frac{\alpha\beta\bar{z}k^\alpha}{1+\alpha\beta}$. And optiam value function is $V^2(k)=\ln(\bar{z}k^\alpha-\frac{\alpha\beta\bar{z}k^\alpha}{1+\alpha\beta})+\beta\ln(\bar{z}(\frac{\alpha\beta\bar{z}k^\alpha}{1+\alpha\beta})^\alpha)=\alpha(1+\alpha\beta)\ln(k)+\phi_1$, 
		where $\phi_1=\ln(\bar{z})-\ln(1+\alpha\beta)+\beta\ln(\bar{z})+\beta\alpha\ln(\alpha)+\alpha\beta\ln(\beta)-\alpha\ln(1+\alpha\beta)$

		The \textbf{Third} iteration:
		Take $V^2(k)$ into bellman equation:

		\begin{equation*}
			V^3(k)=\max_{k'}\{\ln(\bar{z}k^\alpha-k')+\beta\alpha(1+\alpha\beta)\ln(k')+\beta\phi_1\}
		\end{equation*}	
		Take the first order derivative w.r.t $k'$, we can get the optimal $K'=\frac{\alpha\beta(1+\alpha\beta)\bar{z}k^\alpha}{1+\alpha\beta(1+\alpha\beta)}$, and plug $k'$ into $V^3(\cdot)$ to gain the optimal $V^3(x)=\alpha(1+\alpha\beta(1+\alpha\beta)\ln(k))+\phi_2$, where $\phi_2=\alpha\ln(\bar{z})-\ln(1+\alpha\beta(1+\alpha\beta))+\alpha\beta(1+\alpha\beta)(\ln(\alpha\beta(1+\alpha\beta)\bar{z})-\ln(1+\alpha\beta(1+\alpha\beta)))$.
		Take the \textbf{Fourth} iteration:
		\begin{equation*}
			V^4(k)=\max_{k'}\{\ln(\bar{z}k^\alpha-k')+\beta\alpha(1+\alpha\beta(1+\alpha\beta)\ln(k')+\beta\phi_2)\}
		\end{equation*}
		Take the first order derivative w.r.t. $K'$, And the optimal $k'=\frac{\alpha\beta(1+\alpha\beta(1+\alpha\beta))\bar{z}k^\alpha}{1+\alpha\beta(1+\alpha\beta(1+\alpha\beta))}$, and the optimal value function of $V^4(k)=\alpha(1+\alpha\beta(1+\alpha\beta(1+\alpha\beta)))\ln(k)+\phi_3$, $\phi_3$ is the term that is not revelant to k.
		If the initial guess is different, we may get different functions\footnote{See \url{https://lhendricks.org/econ720/ih1/Dp_ln.pdf}}.
\end{enumerate}







\section{Problem 3}
\textbf{Solution:}

\begin{enumerate}[a.]
	\item Before we set up a bellman equation, we need to rearrange the sequence problem to:
	 \begin{equation*}
		 W=\sum_{t=0}^\infty \beta^t u(f(k_t,n_t)-k_{t+1},1-n_t)
	 \end{equation*}
	 We then set up a bellman equation:
	 \begin{equation*}
		 V(k)=\max_{n,k'}\{u(f(k,n)-k',1-n)+\beta V(k')\}
	 \end{equation*}
	\item We hope to gain the policy function,i.e., a control variable can be expressed by the state variable k. Once We
	 		know n and k, we know c automatically. 
	\item First, we take the first order condition w.r.t n:
	 \begin{equation*}
		 u_c(c,n)(\frac{\partial f(k,n)}{\partial n}-\frac{\partial f(k,n)}{\partial n})+u_n(c,n)(-1)+\beta V'(k')\frac{\partial f(k,n)}{\partial k}=0
	 \end{equation*}
	 Second, we use Envelop Theorem:
	 \begin{equation*}
		 V'(k)=u_c(c,n)\frac{\partial f(k,n)}{\partial k}
	 \end{equation*}
	\item Back to our sequence problem, and set up a lagrangian equation:
	 	\begin{equation*}
			 \mathcal{L}:=\sum_{t=0}^\infty \beta^t u(c_t,1-n_t)+\lambda_t(f(k_t,n_t)-c_t-k_{t+1})
		 \end{equation*}
		 Take first order derivative w.r.t c and n respectively.
		 \begin{equation*}
			 \beta^tu_c(c_t,1-n_t)=-\lambda_t
		 \end{equation*}
		 \begin{equation*}
			 \beta^tu_n(c_t,1-n_t)(-1)=\lambda_t f_n(k_t,n_t)
		 \end{equation*}
		 Thus, we can get:
		 \begin{equation*}
			\frac{MU_c}{MU_n}=\frac{1}{f_n(k_t,n_t)} 
		 \end{equation*}
	\item Replace value function with the equation we derived in envelop theorem:
		 \begin{equation*}
			 u_n(c,n)=\beta\frac{\partial f(k,n)}{\partial k}u_c(c',n')\frac{\partial  f(k',n')}{\partial k}
		 \end{equation*}
\end{enumerate}






































\end{document}