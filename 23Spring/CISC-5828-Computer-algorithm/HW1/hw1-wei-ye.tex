%EX TS-program = pdflatex
% !TEX encoding = UTF-8 Unicode

% This is a simple template for a LaTeX document using the "article" class.
% See "book", "report", "letter" for other types of document.

\documentclass[11pt]{article} % use larger type; default would be 10pt
\usepackage[utf8]{inputenc} % set input encoding (not needed with XeLaTeX)

%%% Examples of Article customizations
% These packages are optional, depending whether you want the features they provide.
% See the LaTeX Companion or other references for full information.
\usepackage{amsmath}
\makeatletter
\renewcommand*\env@matrix[1][*\c@MaxMatrixCols c]{%
	\hskip -\arraycolsep
	\let\@ifnextchar\new@ifnextchar
	\array{#1}}
\makeatother

\newcommand{\norm}[1]{\left\lVert#1\right\rVert}
%%% PAGE DIMENSIONS
\usepackage{geometry} % to change the page dimensions
\usepackage{listings}
\usepackage[dvipsnames]{xcolor}
\usepackage{marvosym}
\geometry{a4paper} % or letterpaper (US) or a5paper or....
% \geometry{margin=2in} % for example, change the margins to 2 inches all round
% \geometry{landscape} % set up the page for landscape
%   read geometry.pdf for detailed page layout information

\usepackage{graphicx} % support the \includegraphics command and options
% \usepackage[parfill]{parskip} % Activate to begin paragraphs with an empty line rather than an indent
\usepackage{amssymb}
\usepackage{natbib}
\usepackage{hyperref}
\usepackage{mathrsfs}
%%% PACKAGES
\usepackage{booktabs} % for much better looking tables
\usepackage{array} % for better arrays (eg matrices) in maths
\usepackage{paralist} % very flexible & customisable lists (eg. enumerate/itemize, etc.)
\usepackage{verbatim} % adds environment for commenting out blocks of text & for better verbatim
\usepackage{subfig} % make it possible to include more than one captioned figure/table in a single float
% These packages are all incorporated in the memoir class to one degree or another...
\usepackage{pgfplots}
%%% HEADERS & FOOTERS
\usepackage{fancyhdr} % This should be set AFTER setting up the page geometry
\pagestyle{fancy} % options: empty , plain , fancy
\renewcommand{\headrulewidth}{0pt} % customise the layout...
\lhead{}\chead{}\rhead{}
\lfoot{}\cfoot{\thepage}\rfoot{}

%%% SECTION TITLE APPEARANCE
\usepackage{sectsty}
\allsectionsfont{\sffamily\mdseries\upshape} % (See the fntguide.pdf for font help)
% (This matches ConTeXt defaults)
\usepackage[thinc]{esdiff}
\usepackage{bbold}
\usepackage{MnSymbol,wasysym}
%%% ToC (table of contents) APPEARANCE
\usepackage[nottoc,notlof,notlot]{tocbibind} % Put the bibliography in the ToC
\usepackage[titles,subfigure]{tocloft} % Alter the style of the Table of Contents
\renewcommand{\cftsecfont}{\rmfamily\mdseries\upshape}
\renewcommand{\cftsecpagefont}{\rmfamily\mdseries\upshape} % No bold!

%%% END Article customizations

%%% The "real" document content comes below...

\title{Homework 1}
\author{Wei Ye\footnote{2nd year PhD student in Economics Department at Fordham University. Email: wye22@fordham.edu}
    \\ CISC5825 - Computer Algorithm}
\date{Due on Feb 6, 2023}
\begin{document}
\maketitle

\section*{Problem1}
Express function below in terms of growth functions using the best fit.
\begin{enumerate}[a)]
    \item $2^n + n^2 + n$
    \item $(\frac{n+3}{5})^2$
\end{enumerate}
\textbf{Solution:}
\begin{enumerate}[a)]
    \item Since $2^n >= n^2$ in general, thus, $2^n +n^2 + n \leq 2^n + 2^n + 2^n = 3 \cdot 2^n$. $c_1 = 3$, $g(n)= 2^n$, and $n_0 = 1$. The big O notion is $O(2^n)$.
    \item \begin{align*}
        (\frac{n+3}{5})^2 &= \frac{n^2 + 6n + 9}{25}\\
                         &= \frac{n^2}{25} + \frac{6n}{25} + \frac{9}{25}
    \end{align*}
    Thus,
    \begin{equation*}
        \frac{n^2}{25}\leq f(n) \leq \frac{n^2}{25} + \frac{6n^2}{25} + \frac{9n^2}{25}= \frac{16}{25}n^2
    \end{equation*}
    $c_1 = \frac{1}{25}$, $c_2 = \frac{16}{25}$, $n_0 = 1$, and $g(n) = n^2$. Thus it's $\Theta(n^2)$
\end{enumerate}




\section*{Problem 2}
Determine which of the two functions in the pair of of the function below grows faster: $(n)^{\sqrt{n}}$ vs $(\sqrt{n})^n$

\textbf{Solution:}

Make monotone transformation for each of the functions. $\lg(n^{\sqrt{n}})= \sqrt{n}\lg n$, $\lg ((\sqrt{n})^n)= n\lg \sqrt{n}$. Make monotone transformation again,
$\lg \sqrt{n}\lg n = \lg (\sqrt{n}\lg n) = \lg \sqrt{n} + \lg\lg n = \frac{1}{2}\lg n + \lg \lg n$, $\lg n\lg \sqrt{n} = \lg n + \lg \frac{1}{2}\lg n$. 

WLOG let $\lg n = Z$, thus compare $\frac{1}{2}Z + \lg Z $ and $Z+ \lg \frac{1}{2}Z$

Make substraction: $Z+ \lg \frac{1}{2}Z -(\frac{1}{2}Z + \lg Z) = \frac{1}{2}Z +\lg \frac{\frac{1}{2}Z}{Z} = \frac{1}{2}Z +\lg \frac{1}{2} = M$.

When $M \geq 0 \rightarrow \frac{1}{2}Z \geq -\lg \frac{1}{2}  \rightarrow Z \geq \lg (\frac{1}{2})^{-2} = \lg 4 \rightarrow \lg n \geq \lg 4$

When $n \geq 4 $, $(\sqrt{n})^n \geq (n)^{\sqrt{n}}$, otherwise, $(\sqrt{n})^n < (n)^{\sqrt{n}}$

Thus, $(\sqrt{n})^n$ grows faster.

\end{document}